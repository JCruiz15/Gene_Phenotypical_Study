\section{Discusión}

Como podemos observar en la figura \ref{fig:dendrogram} de los resultados nos hemos centrado en el cluster número 2/6 debido a la calidad de los resultados obtenidos tras aplicar a los genes un enriquecimiento funcional. 

\subsection{Comunidad de color gris??}

La comunidad 2 como bien podemos observar en la tabla \ref{tab:comunity2} muestra los genes más sobre-expresados del grupo de genes que estamos estudiando. Estos genes están ordenados según su p-valor asociado, que es una medida estadística que nos ayuda a determinar como de significativos son los resultados obtenidos. Un p-valor menor a 0.05 determina rechaza la hipótesis nula. Como podemos observar en nuestros resultados todos superan este umbral por lo que se concluye que hay evidencia estadística. 

A continuación fijándonos en la columna de N genes/total podemos ver como en algunos de ellos encontramos algunos genes bastantes mas significativos que otros concretamente la fila número 6 de la tabla \ref{tab:comunity2} donde tenemos dos genes sobre-expresados en un proceso en el que participan 26 genes en total, por lo que hemos buscado la función biológico de estos genes y el término de GO del proceso.

\subsubsection{\textbf{Hablar de las funciones biologicas del RNASEH2A y TREX1} }