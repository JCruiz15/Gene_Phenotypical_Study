\section{Conclusiones}

Tal y como pretendíamos en \ref{objetivos} hemos realizado un estudio acerca de los procesos moleculares obtenidos unos resultados muy interesantes de los genes relacionados con el fenómeno de Raynaud como vemos en las tablas de resultados \ref{resultados}. No obstante como ya sabemos el estudio de las enfermedades raras contempla algunas limitaciones debido a la complejidad de las interacciones y los procesos que ocurren en la enfermedad así como la variedad de mutaciones genéticas y combinaciones distintas que dan lugar a las enfermedades raras.

Respecto a nuestro estudio somos conscientes de la falta de datos de seguimiento de pacientes ya que pese a tener multitud de datos a nivel biológico y molecular, entendemos que este tipo de enfermedades se pueden expresar de diferentes maneras según el paciente y puede ser beneficioso buscar terapias a nivel de individuo. \\
\\
Como futuras líneas de investigación creemos que lo más importante sería el desarrollo de nuevas terapias génicas para tratar los síntomas subyacentes al síndrome de Raynaud y un estudio de seguimiento más a largo plazo que nos permita evaluar la efectividad de las terapias y detectar cambios en la condición del paciente y la evolución de la patología. Otra línea para continuar el estudio de este proyecto sería la investigación del catalizador que produce la enfermedad analizando a pacientes de temprana edad para hallar con precisión el origen del fenómeno.