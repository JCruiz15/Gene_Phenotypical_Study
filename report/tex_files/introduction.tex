\section{Introducción}

\subsection{Ontologías}

Para poder estudiar los sistemas influidos por el gen HP:0030880 será necesario utilizar diferentes ontologías y bases de datos que podemos encontrar en Internet. Para ello en primer lugar debemos conocer qué es una ontología. Se podría definir como una manera formal de representar el conocimiento colectivo sobre un área en la que sus conceptos se describen por su significado y su relación entre el resto de componentes \cite{WhatIsOntology}\cite{BioOntologies}.

En nuestro caso, usaremos la ontología de GO (\textit{Gene Ontology} \ref{GO}) y la base de datos HPO (\textit{Human Phenotype Ontology database} \ref{HPO}) \cite{HPOweb}. Dicha  ontología se encarga de mostrar las relaciones entre genes de diversas especies y HPO almacena ontologías asociadas a fenotipos de enfermedades del ser humano.

\subsection{Fenómeno Raynaud}

En particular el gen HP:0030880 se asocia al fenómeno de Raynaud, esta enfermedad afecta al flujo de sangre en vasos arteriosos cuando baja la temperatura corporal, esto suele afectar en su mayoría a las extremidades superiores e inferiores.

Más concretamente, el fenómeno de Raynaud causa una vasoconstricción de los vasos sanguíneos en situaciones de bajas temperaturas o estrés. Cuando esto ocurre, la sangre no puede llegar a la superficie de la piel y las áreas afectadas se vuelven blancas y azules. Cuando el flujo sanguíneo regresa, la piel se enrojece y se tiene una sensación de palpitación o de hormigueo. En casos severos, la pérdida del flujo sanguíneo puede causar llagas o muerte de los tejidos.\cite{RaynaudNIH}

\subsubsection{Tipos de Raynaud}

Es importante conocer que el fenómeno se divide en dos tipos de expresión en función del origen del mismo.
\begin{itemize}
	\item \textbf{Raynaud primario} o \textbf{enfermedad de Raynaud}: Es la forma más frecuente y a su vez la que provoca síntomas más leves, típicamente afectan a mujeres menores de 30 años (normalmente en la adolescencia) de momento no se conoce su origen, ya que el fenómeno de Raynaud primario ocurre en pacientes sin otra enfermedad reumática. \cite{RaynaudFen}
	
	\item \textbf{Raynaud secundario}: Es menos común y se denomina así ya que es provocado por otra condición. Estas son diversas y entre las diferentes enfermedades reumáticas que lo provocan se incluyen, entre otras, la esclerodermia, el lupus o el síndrome de Sjögren. Además puede deberse también a largas exposiciones al frío o ciertas sustancias químicas. Los síntomas provocados por el Raynaud secundario suelen ser más graves que en el primario. A menudo el fenómeno se presenta en mujeres mayores de los 30 años. \cite{RaynaudClass}
\end{itemize}

\subsection{Niveles de asociación}

\subsection{Objetivos}

En este proyecto vamos a investigar los mecanismos moleculares y rutas biológicas detrás de la enfermedad descrita por el término HP:0030880 y sus genes asociados, además de intentar buscar opciones de terapia útiles para el tratamiento de la enfermedad. Para ello usaremos, como ya hemos comentado, ontologías (GO) y bases de datos como OMIM (\textit{Online Mendelian Inheritance in Man database} \ref{OMIM}) o HPO.

