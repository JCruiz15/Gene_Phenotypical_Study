\section{Introducción}

Para poder estudiar los sistemas influidos por el gen HP:0030880 será necesario utilizar diferentes ontologías y bases de datos que podemos encontrar en internet. Para ello en primer lugar debemos conocer qué es una ontología. Se podría definir como una manera formal de representar el conocimiento colectivo sobre un área en la que sus conceptos se describen por su significado y su relación entre el resto de componentes [cita]. En nuestro caso, usaremos la ontología de \textit{Gene Ontology} (GO) y de \textit{Human Phenotype Ontology} (HPO). Estas ontologías se encargan de mostrar las relaciones entre genes de diversas especies y de las relaciones de los diferentes fenotipos que encontramos en enfermedades que puede padecer el ser humano, respectivamente.

\subsection{Fenómeno Raynaud}

En particular el gen HP:0030880 se asocia al fenómeno de Raynaud, esta enfermedad afecta al flujo de sangre en vasos arteriosos cuando baja la temperatura corporal, esto suele afectar en su mayoría a las extremidades superiores e inferiores.
