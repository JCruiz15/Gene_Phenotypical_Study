\documentclass{bmcart}

% CORRECIONES
% TODO: Revisar todas las referencias
% TODO: IDs ENSEMBL en vez de GeneSymbol (??)
% TODO: Separar las tablas en 2 o 3 para agrandar texto
% TODO: Indicar proteinas SINGLETONS en figure 5
% TODO: Cita de GO incorrecta y STRING no puede ir a ayuda

% TERMINAR
% TODO: Completar el codigo de R
% TODO: Resultados
% TODO: Discusión
% TODO: SETUP, LAUNCH, DOCKER y Diapositivas

%%%%%%%%%%%%%%%%%%%%%%%%%%%%%%%%%%%%%%%%%%%%%%
%%                                          %%
%% CARGA DE PAQUETES DE LATEX               %%
%%                                          %%
%%%%%%%%%%%%%%%%%%%%%%%%%%%%%%%%%%%%%%%%%%%%%%

%%% Load packages
\usepackage{amsthm,amsmath}
\usepackage{graphicx}
%\RequirePackage[numbers]{natbib}
%\RequirePackage{hyperref}
\usepackage[utf8]{inputenc} %unicode support
%\usepackage[applemac]{inputenc} %applemac support if unicode package fails
%\usepackage[latin1]{inputenc} %UNIX support if unicode package fails
\usepackage{hyperref}
\usepackage{setspace}
\usepackage{caption}

\begin{document}

	\begin{frontmatter}
	
		\begin{fmbox}
			\dochead{Research}
			
			%%%%%%%%%%%%%%%%%%%%%%%%%%%%%%%%%%%%%%%%%%%%%%
			%% INTRODUCIR TITULO PROYECTO               %%
			%%%%%%%%%%%%%%%%%%%%%%%%%%%%%%%%%%%%%%%%%%%%%%
			
			\title{Estudio fenotípico del gen HP:0030880 asociado al fenómeno de Raynaud}
			
			%%%%%%%%%%%%%%%%%%%%%%%%%%%%%%%%%%%%%%%%%%%%%%
			%% AUTORES. METER UNA ENTRADA AUTHOR        %%
			%% POR PERSONA                              %%
			%%%%%%%%%%%%%%%%%%%%%%%%%%%%%%%%%%%%%%%%%%%%%%
			
			\author[
			  addressref={aff1},                   % ESTA LINEA SE COPIA IGUAL PARA CADA AUTOR
			  corref={aff1},                       % ESTA LINEA SOLO DEBE TENERLA EL COORDINADOR DEL GRUPO
			  email={jaldanam21@uma.es}   % VUESTRO CORREO ACTIVO
			]{\inits{J.A.M.}\fnm{Jesús} \snm{Aldana}} % inits: INICIALES DE AUTOR, fnm: NOMBRE DE AUTOR, snm: APELLIDOS DE AUTOR
			\author[
			  addressref={aff1},
			  email={juancaruru@uma.es}
			]{\inits{J.C.}\fnm{Juan Carlos} \snm{Ruiz}}
			
			%%%%%%%%%%%%%%%%%%%%%%%%%%%%%%%%%%%%%%%%%%%%%%
			%% AFILIACION. NO TOCAR                     %%
			%%%%%%%%%%%%%%%%%%%%%%%%%%%%%%%%%%%%%%%%%%%%%%
			
			\address[id=aff1]{%                           % unique id
			  \orgdiv{ETSI Informática},             % department, if any
			  \orgname{Universidad de Málaga},          % university, etc
			  \city{Málaga},                              % city
			  \cny{España}                                    % country
		}
		
		\end{fmbox}% comment this for two column layout
		
		\begin{abstractbox}
		
			\begin{abstract} % abstract
			
			%%%%%%%%%%%%%%%%%%%%%%%%%%%%%%%%%%%%%%%%%%%%%%%
			%% RESUMEN BREVE DE NO MAS DE 100 PALABRAS   %%
			%%%%%%%%%%%%%%%%%%%%%%%%%%%%%%%%%%%%%%%%%%%%%%%	
			Esta memoria de investigación se realizó siguiendo la estructura convencional de una revista científica, con el objetivo de divulgar una enfermedad rara que afecta a numerosas personas en nuestro país. Hicimos un estudio acerca de aquellos genes a los que afecta y el fenotipo que producen. Toda la información investigada se organizó en forma de grafo y redes biológicas para su posterior análisis.
						
			This research report was carried out following the conventional structure of a scientific journal, with the aim of disseminating a rare disease that affects many people in our country. We did a study about the genes it affects and the phenotype they produce. All the information investigated was organized in the form of a graph and biological networks for further analysis.
			
			\end{abstract}
			
			%%%%%%%%%%%%%%%%%%%%%%%%%%%%%%%%%%%%%%%%%%%%%%
			%% PALABRAS CLAVE DEL PROYECTO              %%
			%%%%%%%%%%%%%%%%%%%%%%%%%%%%%%%%%%%%%%%%%%%%%%
			
			\begin{keyword}
			\kwd{Raynaud}
			\kwd{gene}
			\kwd{phenotype}
			\kwd{ontology}
			\end{keyword}
		
		
		\end{abstractbox}
	
	\end{frontmatter}
	
	\section{Introducción}

\subsection{¿Qué es una ontología?}
Una ontologia es una cosa que relaciona cosas.
\subsection{Fenómeno Raynaud}
	\section{Materiales y métodos}

Se ha llevado a cabo un estudio fenotípico sobre la población afectada por el fenómeno de Raynaud. Se pretende averiguar la influencia de los diferentes genes asociados a la enfermedad y su grado de importancia en el desarrollo de la misma.

PARRAFO acerca de las herramientas utilizadas y los datos con los que hemos trabajado

\subsection{Obtención de la red}
\subsection{Nodos vecinos}
\subsection{Cálculo de comunidades}
\subsection{Enriquecimiento de las comunidades}
\subsection{Estudio de la funcionalidad}
	\section{Resultados}
\label{resultados}
\begin{spacing}{1}
Todo lo comentado en esta sección se ha realizado haciendo uso de la librería linkcomm de R  \cite{Linkcomm}  y la ayuda de STRINGdb \cite{STRINGdb} . 
Tras la obtención de la red de genes al realizar el mapeo con STRINGdb hemos obtenido un grafo como el que podemos observar en la figura \ref{fig:graph1}. En este grafo podemos observar todos los genes que pertenecen a la red asociada al fénomeno de Raynaud. Esto se ha obtenido a partir de STRINGdb como hemos explicado en el apartado \ref{obtencion_red}.
\end{spacing}

\begin{minipage}{\linewidth}
	\makebox[\linewidth]{
		\includegraphics[width=\textwidth]{figures/Raynaud_genes-graph.png}
	}
	\captionof{figure}{Grafo obtenido al realizar el mapeo de genes}
	\label{fig:graph1}
\end{minipage}

\begin{spacing}{1}
	Seguidamente hemos obtenido las comunidades que conforman nuestra red y a las que posteriormente se le harán los estudios de funcionalidad y podremos comprobar que genes conforman el fenotipo más grave de la enfermedad. Se han obtenido 3 gráficas en las que mostramos información acerca de estas comunidades.
	
	En la figura \ref{fig:dendrogram}, la primera obtenida del flujo de trabajo, se observa un dendrograma en la que, diferenciadas por colores, se pueden visualizar las diferentes comunidades obtenidas. Estas se han obtenido a una altura bastante alta (alrededor del 0.9) y nos indica que tenemos 6 comunidades cuyo agrupamiento más grande tiene 11 genes.
\end{spacing}

\begin{minipage}{\linewidth}
	\makebox[\linewidth]{
		\includegraphics[width=0.7\textwidth]{figures/Raynaud_genes-dendrogram.png}
	}
	\captionof{figure}{Dendrograma de la red mostrando las comunidades}
	\label{fig:dendrogram}
\end{minipage}

\begin{spacing}{1}
La segunda figura que obtenemos del flujo de trabajo sería la número \ref{fig:members}, en esta observamos una matriz de los miembros de cada comunidad. En este caso solo podemos ver los 10 primeros genes de nuestra red y en la matriz se nos muestra las comunidades a las que pertenece cada gen (señalado con un cuadrado en color para cada comunidad a la que pertenece). Además en los márgenes derecho e inferior de la matriz observamos los sumatorios del número de total de comunidades a las que pertenece cada gen y el número de genes que contiene cada comunidad, respectivamente.
\end{spacing}

\begin{minipage}{\linewidth}
	\makebox[\linewidth]{
		\includegraphics[width=0.7\textwidth]{figures/Raynaud_genes-comunity_members_matrix.png}
	}
	\captionof{figure}{Matriz de los primeros genes para cada comunidad}
	\label{fig:members}
\end{minipage}

\begin{spacing}{1}
	Por último, al obtener las comunidades también hemos mostrado la figura \ref{fig:comunity_graph} en la que se observa un grafo similar al de la figura \ref{fig:graph1}, sin embargo, en este tenemos diferenciadas por colores las diferentes comunidades que hemos obtenido tras realizar el flujo de trabajo.
\end{spacing}

\begin{minipage}{\linewidth}
	\makebox[\linewidth]{
		\includegraphics[width=0.7\textwidth]{figures/Raynaud_genes-comunities_graph.png}
	}
	\captionof{figure}{Grafo de los genes divididos por comunidad}
	\label{fig:comunity_graph}
\end{minipage}

\begin{spacing}{1}
Por último, hemos obtenido los CSV de cada una de las comunidades tras aplicarles el enriquecimiento. No todas las comunidades han devuelto una tabla con datos y los datos que sí hemos obtenido podemos observarlos en las tablas \ref{tab:comunity1} y \ref{tab:comunity2}, correspondiendo a las comunidades 1 y 2, respectivamente.

En estas tablas podemos observar las columnas de \textit{Ontología}, que nos indica la ontología de la que se ha obtenido esa referencia; \textit{Término}, que indica el código de la ontología que relaciona a dicho gen; \textit{N genes} y \textit{Genes totales}, que indican el número de genes encontrados y el número de genes que tiene asociado dicho término; en la columna \textit{Genes} vemos los nombres de los genes encontrados; y por último tenemos dos columnas para el \textit{p valor} y el \textit{fdr}, además de una descipción del término de la ontología.
\end{spacing}

\begin{table}[!ht]
	\centering
	\resizebox{\textwidth}{!}{
	\begin{tabular}{|cccccccccc|}
		\hline
		\textbf{Ontología} & \textbf{Categoría} & \textbf{Término} & \textbf{N genes} & \textbf{Genes totales} & \textbf{Taxon Id} & \textbf{genes} & \textbf{p valor} & \textbf{fdr} & \textbf{descripción} \\ \hline
		GO & Process & GO.0032480 & 2 & 43 & 9606 & RNF125,IFIH1 & 5.17e-06 & 0.0011 & negative regulation of type I interferon production \\ 
		GO & Process & GO.0045088 & 2 & 361 & 9606 & RNF125,IFIH1 & 0.00034 & 0.018 & regulation of innate immune response \\ 
		GO & Process & GO.0043900 & 2 & 653 & 9606 & RNF125,IFIH1 & 0.0011 & 0.0347 & regulation of multi-organism process \\ 
		GO & Process & GO.0032446 & 2 & 690 & 9606 & RNF125,IFIH1 & 0.0012 & 0.0347 & protein modification by small protein conjugation \\ \hline
	\end{tabular}
	}
	\vspace{5px}
	\caption{Resultado del enriquecimiento de la comunidad 1}
	\label{tab:comunity1}
\end{table}

\begin{table}[!ht]
	\centering
	\resizebox{\textwidth}{!}{
	\begin{tabular}{|c|ccccccccc|}
		\hline
		\textbf{Ontología} & \textbf{Categoría} & \textbf{Término} & \textbf{N genes} & \textbf{Genes totales} & \textbf{Taxon Id} & \textbf{genes} & \textbf{p valor} & \textbf{fdr} & \textbf{descripción} \\ \hline
		GO & Process & GO.0090305 & 4 & 287 & 9606 & RNASEH2A,SAMHD1,TREX1,RNASEH2B & 2.37e-07 & 4.26e-05 & nucleic acid phosphodiester bond hydrolysis \\ 
		GO & Process & GO.0034655 & 4 & 394 & 9606 & RNASEH2A,SAMHD1,RNASEH2C,RNASEH2B & 8.29e-07 & 7.46e-05 & nucleobase-containing compound catabolic process \\ 
		GO & Process & GO.0090501 & 3 & 137 & 9606 & RNASEH2A,SAMHD1,RNASEH2B & 3.55e-06 & 9.12e-05 & RNA phosphodiester bond hydrolysis \\ 
		GO & Process & GO.0006401 & 3 & 237 & 9606 & RNASEH2A,RNASEH2C,RNASEH2B & 1.79e-05 & 4e-04 & RNA catabolic process \\ 
		GO & Process & GO.0006298 & 2 & 26 & 9606 & RNASEH2A,TREX1 & 1.97e-05 & 4e-04 & mismatch repair \\ 
		GO & Process & GO.0090502 & 2 & 70 & 9606 & RNASEH2A,RNASEH2B & 0.00013 & 0.0024 & RNA phosphodiester bond hydrolysis, endonucleolytic \\ 
		GO & Process & GO.0090304 & 5 & 3941 & 9606 & RNASEH2A,SAMHD1,TREX1,RNASEH2C,RNASEH2B & 0.00033 & 0.0046 & nucleic acid metabolic process \\ 
		GO & Process & GO.0006260 & 2 & 203 & 9606 & RNASEH2A,TREX1 & 0.0011 & 0.01 & DNA replication \\ 
		GO & Process & GO.0016070 & 4 & 3430 & 9606 & RNASEH2A,SAMHD1,RNASEH2C,RNASEH2B & 0.0041 & 0.0318 & RNA metabolic process \\ \hline
	\end{tabular}
	}
	\vspace{5px}
	\caption{Resultado del enriquecimiento de la comunidad 2}
	\label{tab:comunity2}
\end{table}
	\section{Discusión}

Tras realizar nuestro experimento y recavar los resultados, realizaremos una interpretación de los mismos. Nos centraremos en las comunidades 1 y 2, ya que son las que nos han ofrecido mejores resultados, para entender que funcionalidad biólogica desempeñan y ofrecer una posible terapia que ayude a tratar esta enfermedad. También se dará una posible explicación al porqué no hemos obtenido resultados satisfactorios para el resto de comunidades.

Fijándonos en la figura \ref{fig:dendrogram}, en la que podemos observa el dendrograma de las comunidades obtenidas, vemos que las dos comunidades más grandes corresponden a las 1 y a la 2, que son en las que centraremos nuestras conclusiones.

\subsection{Comunidad 2}

La comunidad 2 como bien podemos observar en la tabla \ref{tab:comunity2} muestra los genes más sobre-expresados del grupo de genes que estamos estudiando. Estos genes están ordenados según su p-valor asociado, que es una medida estadística que nos ayuda a determinar como de significativos son los resultados obtenidos. Un p-valor menor a 0.05 determina rechaza la hipótesis nula. Como podemos observar en nuestros resultados todos superan este umbral por lo que se concluye que hay evidencia estadística. 

Si observamos las descripciones de los enriquecimientos funcionales obtenidos, podemos conseguir una imagen general de la función biológica en la que participan los genes de esta comunidad. Observamos que los genes de esta comunidad en su mayoría participan en procesos metabólicos relacionados con la hidrólisis de los puentes de hidrógeno entre los ácidos nucléicos. Esto nos hace pensar que podría tratarse de un conjunto de proteínas encargado de la apertura de las cadenas de ADN en determinados procesos metabólicos.

Fijándonos en la columna de \textit{N genes/total} podemos ver como en algunos de ellos encontramos algunos genes bastantes más significativos que otros concretamente la fila número 6 de la tabla \ref{tab:comunity2} donde tenemos dos genes expresados en un proceso en el que participan 26 genes en total.

Estos dos genes son \textbf{RNASEH2A} y \textbf{TREX1} que participan en varios procesos de la comunidad, como la hidrólisis de enlaces fosfodiéster, la replicación del ADN y la reparación de errores en la duplicación del material genético. Con lo que con seguridad podemos afirmar que estos genes se encargan de la replicación celular.

Con todas estas conclusiones, se observa una importancia demasiado esencial para aplicar cualquier tratamiento de inhibición o relajación de funcionalidad en estos genes, ya que su función es necesaria para la vida del paciente.

\subsection{Comunidad 1}

En la communidad 1 solo se ha obtenido el enriquecimiento de dos genes: \textbf{RNF125} y \textbf{IFIH1}. Los cuales participan, en mayor o menor medida, en diferentes procesos como pueden ser la regulación negativa de la producción de interferones, la regulación de la respuesta inmune nativa o la modificación de proteínas por uniones proteicas.

Concretamente, participan mayormente en el proceso de la regulación de la producción de interferones tipo I. Los interferones tipo I son un subgrupo de proteínas que regulan la actividad del sistema inmunológico interaccionando con linfocitos, macrófagos, firoblastos, etc \cite{Meager2006TheApplication}. Esto parece indicar que estos genes se relacionan con el LES, enfermedad asociada al fenómeno de Raynaud y que provoca una enfermedad auto-inmune.
Es probable que una mutación en estos genes, produzcan una reacción auto-inmune que produzca los síntomas de LES y que estos deriven en el fenómeno de Raynaud.

Esto, nos puede ofrecer una posible vía para idear una terapia ante esta enfermedad. Se puede inhibir este gen mediante fármacos y así evitar la enfermedad auto-inmune, sin embargo esto provacaría una falta de defensas del organismo ante patógenos externos, con lo que sería conveniente obtener estas proteínas de regulación de forma artificial. No obstante, esta terapia sería bastante costosa y peligrosa para el paciente.

\subsection{Calidad de los resultados}

El resto de comunidades, de las que no hemos obtenido resultados tras el enriquecimiento funcional, no ofrecen nuevas conclusiones por la ausencia de datos empíricos. 

Esta falta de datos, puede deberse a varias razones:

\begin{itemize}
	\item 
\end{itemize}
\end{itemize}

	\section{Conclusiones}

Tal y como pretendíamos en \ref{objetivos} hemos realizado un estudio acerca de los procesos moleculares obtenidos unos resultados muy interesantes de los genes relacionados con el fenómeno de Raynaud como vemos en las tablas de resultados \ref{resultados}. No obstante como ya sabemos el estudio de las enfermedades raras contempla algunas limitaciones debido a la complejidad de las interacciones y los procesos que ocurren en la enfermedad así como la variedad de mutaciones genéticas y combinaciones distintas que dan lugar a las enfermedades raras.

Respecto a nuestro estudio somos conscientes de la falta de datos de seguimiento de pacientes ya que pese a tener multitud de datos a nivel biológico y molecular, entendemos que este tipo de enfermedades se pueden expresar de diferentes maneras según el paciente y puede ser beneficioso buscar terapias a nivel de individuo. \\
\\
Como futuras líneas de investigación creemos que lo más importante sería el desarrollo de nuevas terapias génicas para tratar los síntomas subyacentes al síndrome de Raynaud y un estudio de seguimiento más a largo plazo que nos permita evaluar la efectividad de las terapias y detectar cambios en la condición del paciente y la evolución de la patología. Otra línea de estudio para continuar el estudio de este proyecto sería la investigación del catalizador que produce la enfermedad analizando a pacientes de temprana edad para hallar con precisión el origen del fenómeno.
	
	
	%%%%%%%%%%%%%%%%%%%%%%%%%%%%%%%%%%%%%%%%%%%%%%
	%% OTRA INFORMACIÓN                         %%
	%%%%%%%%%%%%%%%%%%%%%%%%%%%%%%%%%%%%%%%%%%%%%%
	
	\begin{backmatter}
	
		\section*{Abreviaciones}%% if any
			\begin{itemize}
				\item \label{GO} \textbf{GO}: Gene Ontology 
				\item \label{HPO} \textbf{HPO}: Human Phenotype Ontology database
				\item  \label{OMIM} \textbf{OMIM}: Online Mendelian Inheritance in Man database
				\item \label{GSEA} \textbf{GSEA}: Gene Sequence Enrinchment Analysis
				\item \label{NASQAR} \textbf{NASQAR}: Nucleic Acid Sequence Analysis Resource
				\item \label{KEGG} \textbf{KEGG}: Kyoto Encyclopedia of Genes and Genomes
				\item \label{ECs} \textbf{ECs}: Endothelial Cells
				\item \label{STAT} \textbf{STAT}: Signal Transducer and Activator of Transcription
			\end{itemize}
		
		\section*{Disponibilidad de datos y materiales}%% if any
			Repositorio de github:
			\href{https://github.com/jesusaldanamartin/Gene_Phenotypical_Study}{https://github.com/jesusaldanamartin/Gene\_Phenotypical\_Study}
		
		\section*{Contribución de los autores}
			J.C. : Encargado de la introducción, Resultados 
			
			J.A.M. : Encargado del abstract, parte de la introducción y Materiales y Métodos
			%OJO: que sea realista con los registros que hay en vuestros repositorios de github. 
		
		
		%%%%%%%%%%%%%%%%%%%%%%%%%%%%%%%%%%%%%%%%%%%%%%%%%%%%%%%%%%%%%%%%%%%%%%%%%%%%%%%%%%%%%%%%
		%% BIBLIOGRAFIA: no teneis que tocar nada, solo sustituir el archivo bibliography.bib %%
		%% por el que hayais generado vosotros                                                %%
		%%%%%%%%%%%%%%%%%%%%%%%%%%%%%%%%%%%%%%%%%%%%%%%%%%%%%%%%%%%%%%%%%%%%%%%%%%%%%%%%%%%%%%%%
		
		\bibliographystyle{bmc-mathphys} % Style BST file (bmc-mathphys, vancouver, spbasic).
		\bibliography{bibliography.bib}      % Bibliography file (usually '*.bib' )
	
	\end{backmatter}
\end{document}
